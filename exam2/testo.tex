







\documentclass[12pt,italian]{article}

%\usepackage[T1]{fontenc}
%\usepackage[latin9]{inputenc}
\usepackage{amsmath}
\usepackage{amssymb}
%\PassOptionsToPackage{normalem}{ulem}
\usepackage{ulem}
\usepackage{babel}

\usepackage{graphicx}


\usepackage[a4paper, total={15cm, 21.5cm}]{geometry}



\usepackage{color}
\usepackage{listings}
\lstset{language=C++,
                basicstyle=\footnotesize\ttfamily,
                keywordstyle=\color{blue}\ttfamily,
                stringstyle=\color{green}\ttfamily,
                commentstyle=\color{red}\ttfamily,
                morecomment=[l][\color{magenta}]{\#}
                }



\newcommand{\be}{\begin{equation}}
\newcommand{\ee}{\end{equation}}
\newcommand{\bea}{\begin{eqnarray}}
\newcommand{\eea}{\end{eqnarray}}

\setlength{\textwidth}{17cm}
\setlength{\oddsidemargin}{-0.5cm}
\setlength{\evensidemargin}{-0.5cm}


\begin{document}







\begin{center}
{\Large\bf Esame di Laboratorio di Fisica Computazionale }\\
{\bf 6 maggio 2014, ore 10.00}

\vskip1truecm

{\large\bf shell scripting}
\end{center}

\begin{enumerate}
\item
Si utilizzi {\tt sed} per modificare il file {\tt modello.txt}\\
1) cancellando le prime tre righe che contengono degli asterischi,\\
2) sostituendo le scritte generiche {\tt cognome} e {\tt nome}
con il vostro nome e cognome.
Si salvi il risultato in un nuovo file.
\end{enumerate}

\clearpage


\begin{center}
{\large\bf Mathematica}
\end{center}

\begin{enumerate}


\item
Si risolva, rispetto alle variabili $(y,z)$, il seguente sistema di equazioni:
\bea
&&x^2+y^2-z=1\nonumber\\
&&z-2 x^2-2 y^2=4
\label{eq:sistema}
\eea

\item
Si disegnino, nello stesso grafico, le due superfici definite dalle due equazioni (\ref{eq:sistema}).


\item
Si risolva la seguente equazione differenziale, parametrica in $\lambda$:
\bea
&&y'(x) + y(x) = e^{-\lambda x} + 3 \sin x\\
&&y(1)=2\nonumber
\label{eq:diff}
\eea

\item
Si disegni in un grafico tridimensionale la soluzione dell'equazione (\ref{eq:diff})
nell'intervallo $x\in [-2,2]$ e con $\lambda\in [-0.5,0.5]$

\item
Si definisca una funzione che esprime il tensore d'inerzia per una particella puntiforme di massa $m$ e di coordinate $X=(x,y,z)$
\be
I= m
\left( 
\begin{tabular}{ccc}
$y^2+z^2$ & $-x y$    & $-x z$ \\
$-x y$    & $x^2+z^2$ & $-y z$ \\
$-x z$    & $-y z$    & $x^2+y^2$
\end{tabular}  
\right)
\ee

\item
Si consideri un sistema di tre particelle puntiformi con le seguenti masse e coordinate:
\bea
m_1 &=& 1,\quad\quad X_1=(4,1,3)\nonumber\\
m_2 &=& 2,\quad\quad X_2=(-1,-1,1)\nonumber\\
m_3 &=& 4,\quad\quad X_3=(1,0,-2)
\eea
e si definisca il tensore d'inerzia totale $I=\sum_{j=1}^3 I_j$.
Si calcolino autovalori $a_i$ e autovettori $\vec v_i$ (con $i=1,2,3$) della matrice $I$.

\item
Si verifichi che la matrice degli autovettori diagonalizza la matrice $I$.


\item
Si consideri l'ellissoide d'inerzia, ovvero la superficie definita dall'equazione
\be
1 = \frac{a_1}{a_{sum}} \rho_x^2 +
    \frac{a_2}{a_{sum}} \rho_y^2 +
    \frac{a_3}{a_{sum}} \rho_z^2 
\ee
dove $a_{sum}=\sum_{j=1}^3 a_j$, con $a_j$ gli autovalori del tensore di inerzia, e dove $\rho_i$ sono delle coordinate cartesiane lungo gli assi principali d'inerzia.\\
Si disegni l'ellissoide nel caso del tensore $I$,
tenendo conto del fatto che i tre assi principali $\vec v_j$ sono ortogonali tra di loro, ma non coincidono con gli assi cartesiani.\\
{\bf LEGGERE il seguente suggerimento:}\\
{\it Come parametrizzazione per il punto nella base $(x,y,z)$ si utilizzi quella seguente:\\
$ (\frac{a_{sum}}{a_1} \sin\theta\cos\phi ,\frac{a_{sum}}{a_2} \sin\theta\sin\phi,
\frac{a_{sum}}{a_3} \cos\theta ) $\\
Per visualizzare l'ellissoide ruotato,
si prenda il primo autovettore (primo asse principale) e si calcolino l'angolo $\theta_1$  che esso forma con l'asse cartesiano $z$ e l'angolo $\phi_1$ che la sua proiezione sul piano $(x,y)$ forma con l'asse $x$; si scrivano quindi le due rotazioni necessarie per ruotare l'asse $z$ nel primo asse principale.\\
Come visto a lezione, per determinare il valore di ciascun punto si possono dare pi\`u istruzioni all'interno del comando {\tt ParametricPlot3D}; si proceda quindi a ruotare ciascun punto, scritto nella base $(x,y,z)$, utilizzando le due matrici di rotazione determinate prima.
 }


\item
Si consideri la seguente mappa (mappa logistica):
\be
x_{n+1}= r x_n (1-x_n)
\label{eq:logistica}
\ee
e si ponga come condizione iniziale $x_0=0.8$.\\
Si scriva una funzione che genera, per $r$ assegnato, i primi 1000 valori della serie.\\
Utilizzando il comando {\tt Take}, si estraggano gli ultimi 32 elementi della lista precedente; con {\tt Union} ci si riduca agli elementi distinti tra questi 32.\\
Con {\tt Map} si generi la lista delle coppie $\{r,y\}$ dove $y$ \`e uno degli elementi ottenuti nel punto precedente.\\
Si effettui una scansione in $r\in [2.4,3.8]$ con passo pari a 0.01, e si generi la lista delle liste di coppie del punto precedente; utilizzando il comando {\tt Flatten[lista,1] } ci si riduca a una sola lista di tutte le coppie $\{r,y\}$,  che pu\`o infine essere visualizzata con {\tt ListPlot}.


\end{enumerate}


\clearpage


\begin{center}
{\large\bf C++}
\end{center}

Si risolva l'esercizio proposto.
Per facilitare la correzione, se possibile includere tutto in un unico file sorgente.
La sufficienza \`e raggiunta risolvendo correttamente
i primi tre punti.



\section*{Esercizio}

\begin{enumerate}

\item
Si scriva una classe {\tt Matrix} che rappresenti una matrice 2x2.
Tra i membri \emph{privati} si pongano le quattro componenti (reali);
tra i membri \emph{pubblici} si scriva un opportuno costruttore
che richieda gli elementi di matrice, con \emph{valori di default}
impostati in modo da realizzare la matrice identit\`a.

\item
%Si scrivano il costruttore di copie e l'operatore di assegnazione.
Si implementi l'operatore ``$*$'' (membro di {\tt Matrix}), che restituisce il prodotto riga per colonna.

\item
Si implementi una funzione membro {\tt Print},
che stampi su \emph{standard output} la generica matrice
nella forma
{
\tt
\begin{tabular}{cc}
a&b\\
c&d
\end{tabular}
}

\item
Si scriva un {\tt main} che istanzi una matrice
identit\`a ed una 
{$m=
\left(
\begin{tabular}{cc}
1&2\\
3&4
\end{tabular}
\right)$
}
e si controlli, stampando i risultati su {\tt cout},
che il prodotto a sinistra (come anche quello a destra) dell'identit\`a
con $M$ sia uguale a $M$.

\item
Tra i membri pubblici, si scriva una funzione membro {\tt Det},
che restituisca il determinante della matrice.

\item
Si scriva una classe {\tt RandomMatrix} che erediti pubblicamente
da {\tt Matrix}.
Il costruttore di default (senza parametri) dovr\`a generare una matrice \emph{simmetrica}
con elementi random reali compresi tra 0 e 1.

\item
Si crei un {\tt std::vector} di \emph{puntatori} a {\tt Matrix},
e lo si riempia con la matrice identit\`a, con la matrice
{$\sigma_x=
\left(
\begin{tabular}{cc}
0&1\\
1&0
\end{tabular}
\right)$
}
e poi con 10 matrici simmetriche random
(tutte allocate \emph{dinamicamente}; ricordarsi di
liberare la memoria allocata, alla fine del programma;
in alternativa usare degli \emph{smart pointer}, se si preferisce).

%\item
%Scrivere una funzione globale {\tt matrix\_compare},
%che richieda due puntatori a {\tt Matrix} e restituisca un {\tt bool}
%(tale funzione \`e detta \emph{predicato}), che prenda il valore
%{\tt true} se il determinante della prima matrice
%\`e minore di quello della seconda, e {\tt false} altrimenti.
%
%\item
%Ordinare il vettore per determinanti crescenti, usando l'algoritmo
%\begin{lstlisting}
%std::sort(it1, it2, pred)
%\end{lstlisting}
%che ordina gli elementi compresi tra gli \emph{iteratori} {\tt it1} (compreso)
%e {\tt it2} (escluso), confrontandoli attraverso il predicato {\tt pred}.
%Infine controllare (stampandoli su {\tt cout}) che
%il primo e l'ultimo elemento siano rispettivamente $\sigma_x$ e l'identit\`a.

\item
Si vuole ora ordinare il vettore per determinanti crescenti, usando l'algoritmo
\begin{lstlisting}
std::sort(it1, it2, pred)
\end{lstlisting}
che ordina gli elementi compresi tra gli \emph{iteratori} {\tt it1} (compreso)
e {\tt it2} (escluso), confrontandoli attraverso il \emph{predicato} {\tt pred}
(si ricorda che un predicato \`e una funzione che restituisce un {\tt bool}).
Si scriva il predicato opportuno, si ordini il vettore, e si controlli infine
(stampandoli su {\tt cout}) che
il primo e l'ultimo elemento siano rispettivamente $\sigma_x$ e l'identit\`a.

\end{enumerate}



\end{document}
